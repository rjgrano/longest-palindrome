\documentclass[12pt,a4paper]{article}

\usepackage[T1]{fontenc}
\usepackage[utf8]{inputenc}
\usepackage[czech]{babel}
\usepackage{xfrac}
\usepackage{color}
\usepackage{listings}

\author{Tomáš Krejčí \\
tomas789@gmail.com}
\title{Nejdelší palindrom}

\definecolor{gray}{rgb}{0.5,0.5,0.5}

\lstset{
	numbers=left,  
	numberstyle=\tiny\color{gray},	
	basicstyle=\scriptsize}

\begin{document}

\maketitle

\section{Zadání}

Úkolem je vyšetřit délku nejdelšího palindromického podřetězce vstupního řetězce v čase lineárním vzhledem k počtu znaků vstupního řetězce.

\section{Definice}

O řetězci řekneme, že je \textbf{palindromický}, pokud se čte z obou stran stejně. To znamená, že je symetrický podle svého středu. Pokud je řetězec liché délky, rozumíme středem prostřední písmeno. V případě sudého řetězce je středem mezera mezi znaky na pozici $\lfloor \frac{n}{2} \rfloor$ a $\lfloor \frac{n}{2} \rfloor + 1$, kde $n$ je délka vstupního řetězce.

V následujícím textu budeme jako \emph{content} označovat vstupní řetězec a \emph{len} bude jeho délka.

Pokud budeme indexovat pole, budeme tak dělat od nuly a to i v textu. To znamená, že v řetězci \textit{ABCD} je prvek na nulté pozici písmeno \textit{A}.

\section{Řešení}

\subsection{Naivní algoritmus}

\label{naivni}

Princip tohoto řešení je vzít všechny možné podřetězce vstupního řetězce a otestovat, zda-li jsou to palindromy.

Vstupní řetězec má celkem ${n \choose 2}$ podřetězců. Vybraný řetězec umíme ověřit, zda je palindromem v lineárním čase vzhledem k délce testovaného podřetězce. Dohromady algoritmus pracuje v čase $O(n^3)$.

Implementace tohoto algoritmu je ukázána v příloze A na straně \pageref{prilohaA}.

\subsection{Kvadratické vylepšení}

Algoritmus popsaný v sekci \ref{naivni} (str. \pageref{naivni}) není příliš vhodný zejména z důvodu, že vybírá všechny možné podřetězce a těch je kvadraticky mnoho.

Možným vylepšením je namísto přes podřetězce iterovat hlavní cyklus algoritmu přes jednotlivé středy palindromů kterých je celkem $2n+1$ a expandovat zkoumanou oblast na obě strany dokud je u řetězce zachována vlastnost býti palindromem.

Jak již bylo řečeno výše, všech možných středů je lineárně mnoho a pro každé ověření nejdelšího palindromu okolo daného středu je potřeba až lineárně mnoho práce v případě překrývajících se palindromů (např. delší posloupnost tvořená jedním znakem). To znamená celkovou složitost $O(n^2)$. Poukažme ještě na skutečnost, že v případě nepřekrývajících se palindromů je složitost lineární.

Implementace algoritmu je opět k nalezení v příloze \ref{kvadraticky} na straně \pageref{kvadraticky}.

\subsubsection{Poznámky k implementaci.} 

Vnější \emph{for} cyklus iteruje přes jednotlivé potenciální středy palindromu. Pokud je proměnná \emph{i} přes kterou se cyklus iteruje sudá, značí to, že aktuálně ukazuje na písmeno. V opačném případě na mezeru mezi písmeny.

Algoritmus pomocí proměnných \emph{ukLevy} a \emph{ukPravy} ukazuje na konce palindromu. 

Před zehájením iterací \emph{while} cyklu je třeba nastavit vhodně počáteční proměnné. Počáteční nastavení ukazuje Obrázek \ref{inicializace-promennych}. Proměnná \emph{curr} obsahuje velikost aktuálního palindromu.

\begin{figure}[h!]
\centering
\begin{tabular}{|c|c|c|c|}
\hline 
 & \emph{curr} & \emph{ukLevy} & \emph{ukPravy} \\ 
\hline 
\textbf{Středem je písmeno} & 1 & $\frac{i}{2} - 1$ & $ukLevy + 2$ \\ 
\hline 
\textbf{Středem je mezera} & 0 & $\frac{i-1}{2}$ & $ukLevy + 1$ \\ 
\hline 
\end{tabular}  
\caption{Inicializace proměnných}
\label{inicializace-promennych}
\end{figure}

\pagebreak

\subsection{Lineární algoritmus (úprava)}

\subsubsection{Algoritmus}

\begin{lstlisting}[language=pascal]
nastav stred aktualniho superpalindromu na prvni pismenko;
nastav velikost palindromu se stredem v prvnim pismenku na 1;
for i := 1 to 2*N+1 do begin
	if stred i je uvnitr superpalindromu then begin
		if leva hrana symetrickeho palindromu presahuje 
		   levou hranu superpalindromu then begin
			nastav velikost palindromu se stredem v i na velikost
			symetrickeho palindromu nalezici do superpalindromu;
		end else begin
			nastav velikost palindromu se stredem v i na velikost
			symetrickeho palindromu;
			continue;
		end;
	end;
		
	while rozsireni palindromu se stredem i je stale palindrom do
		inkrementuj velikost palindromu se stredem v i;
		if je to prvni palindrom, ktery se podival na 
		   novy znak vstupu then begin
			oznac tento palindrom jako novy superpalindrom;
		end;
	begin 
end;

return max(d[1], d[2], ..., d[2*N+1]);
\end{lstlisting}

\subsubsection{Důkaz správnosti}

Předpokládejme, že výše zmíněný kvadratický algoritmus je správný (triviální - ověřuje všechny možnosti). Lineární algoritmus vznikne tak, že předtím, než začneme expandovat okolo středu, podíváme se, zda-li je uvnitř nějakého většího palindromu. Pakliže uvnitř většího palindromu je, podíváne se na symetrický protějšek, který musí být nutně shodný resp. jeho část uvnitř většího palindromu. V případě, že ho přesahuje, tak pokračujeme expanzi. Odebrali jsme z algoritmu tedy pouze kroky, které již byly jednou počítány a správnost zůstane zachována.

\subsubsection{Důkaz časové složitosti}

Algoritmus je lineární vzhledem k počtu znaků vstupního řetězce, protože naždý znak ze vstupu ne nahlédnut maximálně konstantně krát (dvakrát) a každý z prvků pole $d$ je doplněn právě jednou.


%\subsection{Lineární algoritmus}

%Nejprve zaveďme značení hodnova $d[s]$ bude značit délku nejdelšího palindromu se středem v bodě (písmenu resp. mezeře) $s$. Pod pojmem \emph{superpalindrom} budeme rozumět palindrom překrývající aktuálně zkoumaný palindrom. Takový bude v každou chvíli právě jeden (průběžně se bude měnit).

%Pointou tohoto algoritmu je pouze odstranění redundance z předchozího algoritmu. 

%Předpokládejme, že jsme již kolem středu $s$ našli palindrom délky $d[s]$. Budeme mu říkat superpalindrom. Když algoritmus skončí s prohledáváním tohoto středu, přesune ukazatel na aktuální střed o jednu pozoci dále a opět se pokusí expandovat co možná největší palindrom. Nyní předpokládejme, že jsme o $i$ středů vzdálení od středu superpalindromu s pozice $s+i$ (tedy poloha aktuálního středu) se nachází uvnitř superpalindromu. Nyní mohou nastat dvě situace:

%\begin{itemize}
%\item \emph{Největší palindrom okolo pozice $s+i$ se nechází celý v superpalindromu.}
%V tomto případě se nám stačí podívat, jak dopadl jeho protějšek symetricky dle středu superpalindromu. Nastavíme tedy $d[s+i] := d[s-i]$ (hodnota $d[s-i]$ je v tuto chvíli již známá).
%\item \emph{Největší palindrom okolo pozice $s+i$ se dotýká nebo překračuje hranici superpalindromu.} Tento palindrom nemusíme prohledávat od začátku, ale začneme s takovou velikostí, aby se palindrom dotýkal pravého okraje superpalindromu a pokračujeme v expanzi. Takovýto palindrom v další iteraci algoritmu označíme jako nový superpalindrom.
%\end{itemize}

%Délku nejdelšího superpalindromu zjistíme tak, že projdeme celé pole $d$ a nalezneme v něm největší hodnotu, která je požadovaným výsledkem.

%Uvedeným postupem jsme odstranili skrytou nadbytečnost v kvadratickém algoritmu. Tento postup je validní, protože jsme neodebrali žádný případ, který byl vyšetřován v kvadratickém algoritmu. Pouze jsme chytře využili již známých údajů.

%Algoritmus pracuje v lineárním čase vzhledem k délce vstupního řetězce protože každý znak vstupu vyšetřuje maximálně konstantně-krát (dvakrát) a protože každý prvek pole $d$ vyplňujeme právě jednou.

\appendix

\pagebreak

\section{Naivní algoritmus}
\label{prilohaA}

\begin{lstlisting}[language=c]
#include <stdio.h>
#include <string.h>

#define MAX_INPUT 500

int main(int argc, const char * argv[])
{
    char content[MAX_INPUT];
    FILE * pFile;
    size_t len;
    int max = 0; 
    int i, j, ukLevy, ukPravy;
    
    if (2 != argc)
    {
        printf("Usage: palindrome <filename>");
        return 1;
    }
                
    pFile = fopen(argv[1], "r");
    fscanf(pFile, "%s", content);
        
    len = strlen(content);
    
    for (i = 0; i < len; i++)
    {
        for (j = i; j < len; j++)
        {
            ukLevy = i;
            ukPravy = j;
            
            while (content[i] == content[j])
            {
                if (ukPravy - ukPravy <= 1)
                {
                    if (j-i > max) max = j-i;
                    break;
                }
                else ++ukLevy, ++ukPravy;
            }
        }
    }
    
    printf("%d\n", max);

    return 0;
}

\end{lstlisting}

\pagebreak

\section{Kvadratický algoritmus}

\label{kvadraticky}

\begin{lstlisting}[language=c]
#include <stdio.h>
#include <string.h>

#define MAX_INPUT 500

int isLetter(int i);

int main(int argc, const char * argv[])
{
    char content[MAX_INPUT];
    FILE * pFile;
    size_t len;
    int max;
    int i, curr, ukLevy, ukPravy;
    
    if (2 != argc)
    {
        printf("Usage: palindrome <filename>");
        return 1;
    }
                
    pFile = fopen(argv[1], "r");
    fscanf(pFile, "%s", content);
        
    len = strlen(content);
    max = 0;
    
    /* i%2 == 0 -> je to pismeno */
    /* i%2 == 1 -> je to mezera */
    
    for (i = 0; i < 2*len + 1; i++)
    {
        if (isLetter(i))
        {
            ukLevy = i / 2 - 1;
            ukPravy = ukLevy + 2;
            curr = 1;
        }
        else
        {
            ukLevy = (i - 1) / 2;
            ukPravy = ukLevy + 1;
            curr = 0;
        }
        
        if (curr > max) max = curr;
        
        while (ukLevy >= 0 && ukPravy < len)
        {
            if (content[ukLevy] == content[ukPravy])
            {
                curr += 2;
                --ukLevy;
                ++ukPravy;
                if (curr > max) max = curr;
            }
            else
            {
                break;
            }
        }
    }
    
    printf("%d\n", max);

    return 0;
}

/* Is it letter or space? */
inline int isLetter(int i)
{
    return i % 2 == 0;
}

\end{lstlisting}

\end{document}